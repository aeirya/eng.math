\part{PDE}
\section{سری و تبدیل فوریه}
\subsection*{مقدمه}
در فصل ۱ هدف حل برخی از معادلات دیفرانسیل پاره ای است. برای این منظور سری و تبدیل فوریه به عنوان یک ابزار مهم استفاده می شوند. به همین دلیل، در این بخش ابتدا چند تعریف اساسی و مهم را در زمینه ی معادلات دیفرانسیل پاره ای ارائه می کنیم و سپس با سری و تبدیل فوریه آشنا می شویم. بعدا در بخش های ۲ و ۳ خواهیم دید که سری و تبدیل فوریه چگونه در حل برخی معادلات دیفرانسیل پاره ای به کار می روند.\\
تعریف ۱ : اگر 
$u:\R^n \to \R$
تابعی از متغیر های مستقل
$x_n,...,x_1$
باشد، یک معادله بین 
$u$
و 
$x_i$
 ها و مشتقات جزئی
 $u$
 نسبت به
 $x_i$
 ها را یک معادله دیفرانسیل با مشتقات جزئی یا معادلات دیفرانسیل پاره ای می نامیم.\\
 به عنوان نمونه

 \begin{equation}
u+u_x+u_{xy}+uu_{yy}=x^2y
\end{equation}
\begin{equation}
u_{xx}+u_{yy}=0
\end{equation}
\begin{equation}
u_t=u_{xx}+uu_x
\end{equation}
\begin{equation}
u_x+u_{xxy}=0
\end{equation}

معادلات دیفرانسیل با مشتقات جزئی هستند.\\
تعریف ۲: مرتبه معادله دیفرانسیل پاره ای عبارت از بزرگترین مرتبه مشتق جزئی ظاهر شده در آن است. به عنوان نمونه معادله دیفرانسیل پاره ای (۴) یک معادله ی مرتبه ۳ است.\\
مرتبه نسبت به هر متغیر مستقل به طور مجزا نیز تعریف می شود. مثلا معادله ی 
$(4)$
 نسبت به 
$x$
مرتبه ۲ و نسبت به 
$y$
مرتبه ۱ است.\\
تعریف ۳ : معادله دیفرانسیل پاره ای را همگن می گوییم اگر جمله ای که فقط وابسته به متغیر های متسقل است، نداشته باشد(برابر با صفر باشد).\\
معادله
$(1)$
  ناهمگن و معادلات
$(4),(3),(2)$
 همگن هستند.\\
 تعریف ۴ : یک معادله دیفرانسیل پاره ای را خطی می گوییم اگر ضریب
 $u$
 و مشتقات جزئی آن فقط تابعی از متغیر های مستقل باشند. معادلات
 $(2)$
 و
 $(4)$
 خطی و معادلات 
 $(1)$
 و
 $(3)$
 به ترتیب به خاطر وجود جملات
 $uu_{yy}$
 و
 $uu_x$
 غیر خطی هستند.\\
 در درس ریاضی مهندسی چند نمونه از معادلات خطی (ضریب ثابت) را بررسی می کنیم.\\
 این معادلات عبارتند از :
 \begin{enumerate}
 	\item 
 	معادله حرارت : 
 	$u_t=\alpha u_{xx}$
 	\item
 	معادله موج (مدل سازی نوسان در یک طناب) : 
 	$u_{tt}=c^2u_{xx}$
 	\item
 	معادله ی لاپلاس (انتقال حرکت پایا در یک صفحه) : 
 	$u_{xx}+u_{yy}=0$
 	\item
 	معادله ی تیر :
 	$u_{tt}=au_{xxxx}$
 \end{enumerate}
نکته : هر مساله ی 
PDE
برای داشتن جواب یکتا، نسبت به متغیر به تعداد مرتبه آن شرط اولیه نیاز دارد.\\
به عنوان نمونه مسئله ی حرارت با داشتن یک شرط اولیه برای 
$t$
و دو شرط مرزی برای 
$x$
جواب یکتا دارد.\\
در ادامه ابتدا سری فوریه و سپس تبدیل فوریه را می بینیم.
\subsection*{مدل سازی مسائل حرارت، لاپلاس و موج}
\subsubsection{انتقال حرارت در یک میله ی فلزی}
todo : pic\\
داریم
\[(\left.KA\frac{\partial u}{\partial x}\right|_{x+\Delta x}-\left.KA\frac{\partial u}{\partial x}\right|_{x})\Delta t=
	mc_p\Big(u(x,t+\Delta t)-u(x,t)\Big)\]
و
\[m=\rho \Delta v=\rho A\Delta x\]
پس
\begin{align*}
\begin{cases}
\left.\frac{\partial u}{\partial x}\right|_{x+\Delta x}-\left.\frac{\partial u}{\partial x}\right|_{x} = \frac{\partial^2 u}{\partial x^2} \Delta x\\
u(x,t+\Delta t)-u(x,t)=\frac{\partial u}{\partial t}\Delta t
\end{cases} 
& \Rightarrow 
\\
KA\frac{\partial^2 u}{\partial x^2}\Delta x\Delta t=m\rho A\frac{\partial u}{\partial t}\Delta t\Delta x
& \Rightarrow
\\
\frac{\partial u}{\partial t}=\frac{k}{m\rho}\frac{\partial^2 u}{\partial x^2}
\end{align*}

پس اگر تعریف کنیم
$\alpha := \frac{k}{m\rho}$
متوجه می شویم که\\
\[u_t=\alpha u_{xx}\]
\[u(0,t)=u(L,t)=0\]
\[u(x,0)=f(x)\]
اگر هم این مسئله را در ۲ بعد بررسی کنیم، خواهیم داشت\\
todo : pic
\[\frac{\partial u}{\partial t}=\alpha(\frac{\partial^2 u}{\partial x^2}+\frac{\partial^2 u}{\partial y^2})\]
\subsubsection{
	معادله لاپلاس، انتقال حرارت پایا
	 \lr{(steady state)}
}
\begin{equation*}
	\begin{aligned}
	{} &\ \frac{\partial u}{\partial t}=0 \\
	&\ \frac{\partial u}{\partial t}=\alpha(\frac{\partial^2 u}{\partial x^2}+\frac{\partial^2 u}{\partial y^2}) \\
	&\ \Rightarrow \frac{\partial^2 u}{\partial x^2}+\frac{\partial^2 u}{\partial y^2}=0
	\end{aligned}
\end{equation*}
پس داریم
\begin{equation*}
\begin{aligned}
{} &\ 
u_{xx}+u_{yy}=0
\\
&\
u_x(0,y)=u_x(a,y)=0
\\
&\
u(x,0)=f(x)
\\
&\
u(x,b)=g(x)
\\
\end{aligned}
\end{equation*}



\subsubsection{موج (نوسان طناب)}
pic : todo\\
با توجه به اصل
Hooke
 داریم
 \[
 F=k[u(x+2\Delta x)-u(x+\Delta x)]-k[u(x+\Delta x)-u(x)]=ma=m\frac{\partial^2 u}{\partial t^2}
 \]
 todo : pic\\
 داریم 
 $L=n\Delta x$
 و
 $M=nm$
 و
 $K=\frac{k}{n}$
 حال با توجه به رابطه قبلی داریم
 \begin{equation*}
 \begin{aligned}
 k\Big(u(x+{}&\ 2\Delta x,t)- 2u(x+\Delta x,t)+u(x,t)\Big)=m\frac{\partial^2 u}{\partial t^2}\\
 &\  \Rightarrow  u_{tt}=\frac{k}{m}u_{xx}(\Delta x)^2\\
 &\  \Rightarrow u_{tt}=\frac{k(\Delta x)^2}{m}u_{xx}\\ 
 &\ \Rightarrow u_{tt}=\frac{nK\frac{L^2}{K^2}}{\frac{M}{n}}u_{xx}\\
 &\  \Rightarrow u_{tt}=\frac{KL^2}{M}u_{xx} \\
 \end{aligned}
 \end{equation*}
تعریف می کنیم
$c^2 = \frac{KL^2}{M}$
.
پس داریم

\begin{equation*}
\begin{aligned}
{}&\ u_{tt}=c^2u_{xx} \\
&\ u(0,t)=u(L,t)=0 \\
&\ u(x,0)=f(x) \\
&\ u_t(x,0)=g(x) \\
\end{aligned}
\end{equation*}

اگر هم این مسئله را در ۲ بعد بررسی کنیم، خواهیم داشت\\
todo : pic\\
\[
u_{tt}=c^2(u_{xx}+u_{yy})
\]

\subsection*{سری فوریه}
۱) تابع پیوسته و متناوب
$f$
 با دوره تناوب
 $2\pi$\\
 اگر 
 $f:\R\to\R$
 پیوسته و متناوب با دوره ی تناوب 
 $2\pi$
 باشد، ضرایب
 $a_k,a_0$
 و
 $b_k$
 موجودند به طوری که به ازای هر
 $x\in\R$
 داریم
 \begin{equation}
 f(x)=\int_{-\pi}^\pi{\frac{a_0}{2}dx}+\sum_{k=1}^\infty {\int_{-\pi}^\pi{a_kcos(kx)dx}}+\sum_{k=1}^\infty{\int_{-\pi}^\pi{b_ksin(kx)dx}}
 \end{equation}
 همچنین داریم
 \begin{equation}
 a_0=\frac{1}{\pi}\int_{-\pi}^\pi {f(x)dx}
 \end{equation}
 \begin{equation}
 a_k=\frac{1}{\pi}\int_{-\pi}^\pi {f(x)cos(kx)dx}
 \end{equation}
  \begin{equation}
 b_k=\frac{1}{\pi}\int_{-\pi}^\pi {f(x)sin(kx)dx}
 \end{equation}
 اثبات - برای محاسبه ضریب
 $a_0$
 ، کافی است از طرفین 
 $(5)$
 در بازه ی
 $[-\pi,\pi]$
 انتگرال بگیریم : 
 \[
 \int_{-\pi}^\pi {f(x)dx}=\int_{-\pi}^\pi{\frac{a_0}{2}}dx+\sum_{k=1}^\infty a_k{int_{-\pi}^\pi{cos(kx)dx}}+\sum_{k=1}^\infty b_k{int_{-\pi}^\pi{sin(kx)dx}}
 \]
 همچنین داریم :
 \[
 \int_{-\pi}^\pi{cos(kx)dx}=\left.\frac{sin(kx)}{k}\right |_{-\pi}^\pi=0
 \]
 \[
  \int_{-\pi}^\pi{sin(kx)dx}=0
 \]
 دقت کنید که دلیل رابطه آخر این است که
 $sin(kx)$
 تابعی فرد است و بازه انتگرال گیری متقارن است. بنابراین داریم
 \[
  f(x)=\frac{a_0}{2}(2\pi)+0+0\Rightarrow  a_0=\frac{1}{\pi}\int_{-\pi}^\pi {f(x)dx}
 \]
 حال برای محاسبه ضرایب 
 $a_j$
 ، ابتدا طرفین تساوی 
 $(5)$
 را در 
 $cos(jx)$
 ضرب می کنیم و سپس در بازه ی 
 $[-\pi,\pi]$
 انتگرال می گیریم :
 \begin{equation*}
 \begin{aligned}
 \int_{-\pi}^\pi{f(x)cos(jx)dx} {} &\ =\frac{a_0}{2}\int_{-pi}^\pi{cos(jx)dx}+\sum_{k=1}^\infty{\int_{-\pi}^\pi{cos(kx)cox(jx)dx}} \\
 &\ +\sum_{k=1}^\infty{a_k\int_{-\pi}^\pi{cos(kx)cox(jx)dx}} \\
 &\ +\sum_{k=1}^\infty{b_k\int_{-\pi}^\pi{cos(kx)sin(jx)dx}} \\
 \end{aligned}
 \end{equation*}
 از طرفی برای هر عدد صحیح
 $j\ne0$
 داریم
 \[
 \int_{-\pi}^\pi{cos(jx)dx}=\left.{\frac{sin(kx)}{k}}\right |_{-\pi}^\pi=0
 \]
 وبرای هر 
 $j,k$
 صحیح، با توجه به فرد بودن تابع
 $sin(kx)cos(jx)$
 داریم
 \[
 \int_{-\pi}^\pi{sin(kx)cos(jx)dx}=0
 \]
 و همچنین داریم
\begin{align*}
 \int_{-\pi}^\pi{cos(kx)cox(jx)dx}&=
\begin{cases}
\int_{-\pi}^\pi{\frac{1}{2}\Big[cos\big((k+j)x\big)+cos\big((k-j)x\big)\Big]} &\mbox{if } k\ne j\\
\int_{-\pi}^\pi{\frac{1+cos(2jx)}{2}dx}   
	&\mbox{if } k=j\\
\end{cases}
\\
&=\begin{cases}
0 &\mbox{if } k\ne j
\\
\pi &\mbox{if } k=j
\end{cases}
\end{align*}
 و بنابراین متوجه می شویم که برای هر 
 $j\in\N$
 داریم
 \[
 \int_{-\pi}^\pi{f(x)cos(jx)dx} {} = 0+a_j\times\pi+0\Rightarrow a_j=\frac{1}{\pi}\int_{-\pi}^\pi{f(x)cos(jx)dx}
 \]
 به طور مشابه با ضرب طرفین
 $(5)$
 در
 $sin(jx)$
 و انتگرال گیری در بازه
 $[-\pi,\pi]$
 می توان نشان داد که برای هر 
 $j\in\N$
 ،
 $b_j=\frac{1}{\pi}\int_{-\pi}^\pi{f(x)sin(jx)dx}$
 است.\\
 مثال- سیگنال مثلثی
\begin{align*}
 &f(x)=
 \begin{cases}
 \pi-x &\mbox{if  } { 0\le x\le \pi}
 \\
 \pi+x &\mbox{if  }  {-\pi\le x \le 0}
 \end{cases}
 &f(x+2\pi)=f(x)
\end{align*}

 حل-\\
 \begin{equation*}
 \begin{aligned}
 a_0 {} &\ = \frac{1}{\pi}\int_{-\pi}^\pi{f(x)dx}\\
 &\ =\frac{1}{\pi}\left(\int_{-\pi}^0{\pi-x \, dx}+\int_0^\pi{\pi+x \, dx}\right)\\
 &\ =\frac{1}{\pi}\left(
 \left.{\frac{(\pi+x)^2}{2}}\right |_{-\pi}^0
 +
  \left.{\frac{-(\pi-x)^2}{2}}\right |_0^\pi
 \right)\\
 &\ = \frac{1}{\pi}\left(
 \frac{\pi^2}{2}-0+0+\frac{\pi^2}{2}
 \right)= \pi
 \\\\
  a_k {} &\ = \frac{1}{\pi}\int_{-\pi}^\pi{f(x)cos(kx)dx}\\
 &\ = \frac{1}{\pi}\left(
 \int_{-\pi}^{0}{(\pi+x)cos(kx) dx} +\int_{0}^{\pi}{(\pi-x)cos(kx)dx}
 \right)\\
 &\ = \frac{1}{\pi}\left(
 \int_{-\pi}^{0}{xcos(kx) dx} -\int_{0}^{\pi}{(xcos(kx)dx}
 \right)
 +
 \int_{-\pi}^{\pi}{cos(kx)dx}
 \end{aligned}
 \end{equation*}

 که در آن 
 \begin{alignat*}{3}
 \int_{-\pi}^{0}{xcos(kx)dx}
 = & \left.{\frac{xsin(kx)}{k}}\right |_{-\pi}^{0} & -\int_{-\pi}^{0}{\frac{sin(kx)}{k}dx}
 &= \frac{1-(-1)^k}{k^2}
\\
 \int_{0}^{\pi}{xcos(kx)dx}
 = & \left.{\frac{xsin(kx)}{k}}\right |_{0}^{\pi} & -\int_{0}^{\pi}{\frac{sin(kx)}{k}dx}
 &=\frac{(-1)^k-1}{k^2}
 \\
 \int_{-\pi}^{\pi}{cos(kx)dx}=& 0
 \end{alignat*}

 بنابراین
 \[
 a_k=\frac{1}{\pi}\frac{2}{k^2}\left(1-(-1)^k\right)=
 \begin{cases}
 0 &\mbox{if } k=2n\\
 \frac{4}{\pi(2n-1)^2} &\mbox{if } k=2n-1
 \end{cases}
 \]
 و
\begin{equation*}
	\begin{aligned}
	b_k {} &\ = \frac{1}{\pi}\int_{-\pi}^{\pi}{f(x)sin(kx)dx}\\
	&\ = \frac{1}{\pi}\left(
	\int_{-\pi}^{0}{(\pi+x)sin(kx)dx}+\int_{0}^{\pi}{(\pi-x)sin(kx)dx}
	\right)\\
	&\ = \frac{1}{\pi}\left(\int_{-\pi}^{0}{xsin(kx)dx}-\int_{0}^{\pi}{xsin(kx)dx} \right)+\int_{-\pi}^{\pi}{sin(kx)dx}
	\end{aligned}
\end{equation*}
که
\begin{alignat*}{2}
{} &\ \int_{-\pi}^{0}{xsin(kx)dx}
	=&\left.{-\frac{xcos(kx)}{k}}\right |_{-\pi}^{0} 
	&+\int_{-\pi}^{0}{\frac{cos(kx)}{k}dx}=\frac{\pi(-1)^{k+1}}{k}\\
&\ \int_{0}^{\pi}{xsin(kx)dx}
	=&\left.{-\frac{xcos(kx)}{k}}\right |_{0}^{\pi} 
	&+\int_{0}^{\pi}{\frac{cos(kx)}{k}dx}=\frac{\pi(-1)^{k+1}}{k}\\
&\ \int_{-\pi}^{\pi}{sin(kx)dx}=0
\end{alignat*}

پس 
$b_k=0$
است. پس\\
\[f(x)=\frac{\pi}{2}+\sum_{n=1}^{\infty}{\frac{4}{n(2n-1)^2}}cos(2n-1)x\]
*توجه : نمودار 
$f(x)$
در مثال ۱ بصورت 
todo
است. بنابراین 
$f(x)$
تابعی زوج است و در رابطه 
$(8)$
از زوج بودن 
$f$
و فرد بودن تابع
$sin(kx)$
نتیجه می شود که 
$b_k=0$.
پس توجه به زوج یا فرد بودن سیگنال 
$f(x)$
محاسبات ضرایب سری فوریه را ساده می کند. بعدا در مورد این نکته مفصل صحبت می کنیم.\\
توجه- اگر تابع 
$f(x)$
پیوسته و با دوره تناوب 
$2\pi$
بوده و در نقطه ی 
$x_0$
دارای ناپیوستگی نوع اول باشد، آنگاه سری فوریه ی
$f$
در
$x_0$
به میانگین حد چپ و راست همگراست.\\
مثال۱ : سری فوریه ی تابع
$f(x))$
با ضابطه ی زیر را بدست آورید.
\begin{align*}
&f(x)=
\begin{cases}
0 &\mbox{if } 0\le x\le \pi
\\
\pi &\mbox{if } -\pi\le x \le 0
\end{cases}
& f(x+2\pi)=f(x)
\end{align*}
حل-
\begin{equation*}
\begin{aligned}
{} &\ a_0=\frac{1}{\pi}\int_{-\pi}^{\pi}{f(x)dx}=\frac{1}{\pi}\int_{0}^{\pi}{\pi dx}=\pi\\
&\ a_k=\frac{1}{\pi}\int_{-\pi}^{\pi}{f(x)cos(kx)dx}=\frac{1}{\pi}\int_{0}^{\pi}{\pi cos(kx) dx}=0\\
&\ b_k=\frac{1}{\pi}\int_{-\pi}^{\pi}{f(x)sin(kx)dx}=\frac{1}{\pi}\int_{0}^{\pi}{\pi sin(kx) dx}=\frac{1-(-1)^k}{k}\\
\end{aligned}
\end{equation*}