%_______________mathematical macros:__________________%
\newcommand{\tarc}{\mbox{\large$\frown$}}
\newcommand{\arc}[1]{\stackrel{\tarc}{#1}}
\newcommand{\norm}[1]{\left|\left| #1 \right|\right|}
\newcommand{\on}[2]{\Big|_{#1}^{#2}}
\newcommand{\inn}[1]{\langle #1\rangle}
\newcommand{\LO}[1]{\mathcal{L}(#1)}
\newcommand{\img}{\mathrm{Img }\ }
\newcommand{\pr}[1]{\left(#1\right)}
\newcommand{\R}{\mathbb{R}}
\newcommand{\qed}{\hfill $\blacksquare$}
\newcommand{\ped}{\newline\null\hfill $\square$}
\newcommand{\size}[1]{\left|#1\right|}
\newcommand{\N}{\mathbb{N}}
\newcommand{\CC}{\mathbb{C}}
\newcommand{\conj}[1]{\overline{#1}}
\newcommand{\ol}[1]{\overline{#1}}
\newcommand{\spn}[1]{\mathrm{span}\left(#1\right)}
\newcommand{\pd}{_/%
}
\newcommand{\nn}{\nonumber}
\newcommand{\tw}{\textwidth}
\newcommand{\findent}{\hspace{\parindent}}
\newcommand*\circled[1]{\tikz[baseline=(char.base)]{
		\node[shape=circle,draw,inner sep=2pt] (char) {#1};}}
\newcommand{\myover}[2]{
\overset{\displaystyle #2}{\displaystyle #1}
}
\newcommand{\myunder}[2]{
	\underset{\displaystyle #2}{\displaystyle #1}
}
\def\doubleunderline#1{\underline{\underline{#1}}}
\newcommand{\ubcolored}[4]{
\begingroup
\color{#3}
\underbrace{\color{#1}#2}_{#4}
\endgroup
}
\newcommand{\RN}[1]{%
	\textup{\uppercase\expandafter{\romannumeral#1}}%
}
\newcommand{\set}[1]{\left\{#1\right\}}
\newcommand{\abs}[1]{\left|#1\right|}


\newcolumntype{M}[1]{>{\centering\arraybackslash}m{#1 cm}}
\newcolumntype{N}[1]{>{\flushleft\arraybackslash}m{#1 cm}}


\newcommand{\naabiRule}{$-------------------------------------$%
	\newline}
\newcommand{\dotrule}{
	\foreach \n in {1,...,61}{$.$\hspace{.15cm}}
	\newline
}

%\newcounter{mamad}
%\newcommand{\chap}[1]{
%	\refstepcounter{mamad}
%	\part*{\themamad\ \ #1}
%	\addcontentsline{toc}{part}{\themamad\hspace{0.25 cm} #1}
%	\stepcounter{chapter}
%	\chapter*{#1}
%}

\definecolor{mycolor}{rgb}{0.85, 0.85, 0.85}
\newtcbox{\mybox}{on line,
	colframe=mycolor,colback=mycolor!10!white,
	boxrule=0.5pt,arc=4pt,boxsep=0pt,left=6pt,right=6pt,top=4pt,bottom=4pt}

\newcommand{\chap}[1]{
\stepcounter{chapter}
\chapter*{\thechapter\ \ #1}
\addcontentsline{toc}{chapter}{\thechapter\hspace{0.25 cm} #1}}

\newcommand{\efn}[1]{\LTRfootnote{ #1}}

\newtheorem{definition}{تعریف}


\newenvironment{lemmanum}[1]{
	\textbf{لم #1.}
	\begin{itshape}%
	}
	{\end{itshape}
	\newline\dotrule}

\newenvironment{lemma}{
	\textbf{لم.}%
	\begin{itshape}%
	}
	{\end{itshape}
	\newline\dotrule}

\newenvironment{customEnv}[1]{%
	\noindent\textbf{#1.}
	\begin{itshape}%
	}
	{\end{itshape}%
	‌\hrule\vspace{.5cm}}

\newenvironment{customEnvnum}[2]{%
	\noindent\textbf{#1 #2.}
	\begin{itshape}%
	}
{
	\end{itshape}%
	\newline\dotrule
}

\newenvironment{proof}{\textit{اثبات.}}{\hfill $\blacksquare$\newline\hrule\vspace{.5cm}}


% added by aeirya:
\newtheorem{example}{مثال}
\newtheorem{problem}{مسئله}
%\newtheorem*{example}{حل}
\newenvironment{rcases}
  {\left.\begin{aligned}}
  {\end{aligned}\right\rbrace}

